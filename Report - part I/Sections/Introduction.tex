\section{Introduction}

\label{sec:intro}
 This report presents the design and analysis of a Low Noise Amplifier (LNA) designed to operate in the ISM (Industrial, Scientific and Medical) band. In modern telecom systems, LNAs play a crucial role in dealing with low amplitude signals
 at high frequencies with high data rates. During the development of this project, advanced RF design concepts were explored, including the use of adaptive loops for impedance, stability, gain and noise within the specified frequency range (3-6 GHz). The challenges faced, and the solutions adopted are detailed in this report.

 The aim of this report is to comprehensively document the process of designing and analyzing
 a Low Noise Amplifier (LNA) for operation in the ISM band, with a focus on achieving critical performance specifications. The circuit will have to be designed following certain detailed specifications, starting by designing a suitable polarization network for the transistor, taking into account the effects of encapsulation.

 \subsection{Objectives}
 The main objectives of this report are as follows:
 \begin{itemize}
     \item Design a Low Noise Amplifier (LNA) for operation in a specified frequency in the 3-6 GHz range.
     \item Analyze the performance of the LNA in terms of gain, noise figure, and stability.
     \item Implement adaptive loops for impedance matching, stability, gain, and noise optimization.
     \item Document the design process, challenges faced, and solutions adopted.
     \item Provide a comprehensive analysis of the LNA's performance through simulation and measurement.
     \item Discuss the implications of the design choices made and their impact on the overall performance of the LNA.S
     \item Provide a Python script for the design and simulation of the LNA using the `scikit-rf` library.
 \end{itemize}
 \subsection{Motivation}
 
 The motivation behind this project stems from the increasing demand for high-performance RF components in modern communication systems. As wireless communication continues to evolve, the need for efficient and reliable amplifiers becomes paramount. The project aims to contribute to the field of RF design by providing a detailed analysis and design of an LNA that meets the statement requirements of modern communication systems. The project also serves as a practical application of theoretical concepts learned in the course, bridging the gap between theory and practice in RF design. 

