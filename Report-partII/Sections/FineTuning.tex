\section{Fine Tuning the LNA Design}

In the previous section, theoretical values for the LNA design were obtained and presented in Tables~\ref{tab:initial-vals-cg} and~\ref{tab:initial-vals-cs}. In this section, the design is fine-tuned to meet the required specifications for gain, noise figure (NF), and impedance matching. The tuning process involved adjusting $W$, $L$, and $R$ across the three target technologies.

In the \SI{45}{\nano\meter} case, the initial design used $L = 3L_{\min}$ to improve channel control. However, this resulted in excessive input capacitance, pushing input poles into the signal band and degrading both impedance matching and gain. Reducing $L$ to $L_{\min}$ mitigated this problem but increased the impact of channel-length modulation, requiring a trade-off to preserve performance.


\begin{table}[H]
    \centering
    \footnotesize
    \caption{Tuned values for the CG stage}
    \begin{tabularx}{\textwidth}{>{\centering\arraybackslash}X 
                                >{\centering\arraybackslash}X 
                                >{\centering\arraybackslash}X 
                                >{\centering\arraybackslash}X 
                                >{\centering\arraybackslash}X 
                                >{\centering\arraybackslash}X
                                >{\centering\arraybackslash}X}
        \toprule
        Node & $g_m$ ratio & W & L & $g_m$ (mS) & R ($\si{\ohm}$) & $V_{bias}$ (mV)  \\
        \midrule

        \multirow{1}{*}{350nm}
        &  1:1 & $\SI{205}{\micro\meter}$ & $\SI{350}{\nano\meter}$  & $17.4$ & $310$ & $1500$  \\

        \midrule
        \multirow{1}{*}{65nm}
        & 1:1 & \SI{113}{\micro\meter}  & \SI{65}{\nano\meter} & --- & 280  & 350 \\
        
        \midrule
        \multirow{1}{*}{45nm}
        &  1:1 & \SI{40}{\micro\meter}  & \SI{80}{\nano\meter} & 21 & 300 & 352 \\


        \bottomrule
    \end{tabularx}
    \label{tab:teo-vals-cg}
\end{table}

For the \SI{350}{\nano\meter} node, $W$ was adjusted to refine the transconductance $g_m$ and match the required input impedance. The resistor $R$ was fine-tuned to reduce the Noise Figure without significantly impacting the gain.

In the \SI{65}{\nano\meter} case, $W$ was similarly tuned to satisfy impedance matching. The resistance was increased slightly to enhance gain while maintaining the NF below the allowable limit.

In the \SI{45}{\nano\meter} case, tuning involved reducing both $W$ and $L$ to minimize input capacitance and prevent premature pole formation in the frequency band of interest. While smaller $L$ increases channel-length modulation, the configuration was optimized to balance gain, impedance, and NF. The final $g_m$ value was chosen to recover the desired gain without compromising input matching.

\begin{table}[H]
    \centering
    \footnotesize
    \caption{Tuned values for the CS stage}
    \begin{tabularx}{\textwidth}
        {@{}%  ⟵ trim left padding
         >{\centering\arraybackslash}X
         *{6}{>{\centering\arraybackslash}X}@{}} % ⟵ trim right padding
        \toprule
        Node & Config & W (\si{\micro\meter}) & L (\si{\nano\meter})
             & $g_m$ (mS) & R (\si{\ohm}) & $V_\text{bias}$ (mV) \\
        \midrule
        \multirow{2}{*}{\SI{350}{\nano\meter}}
            & 1:1 & \SI{220}{\micro\meter} & \SI{350}{\nano\meter} & 22.7 & 130 & 750 \\
            & 1:n & \SI{510}{\micro\meter} & \SI{350}{\nano\meter} & 52.7 &  70 & 750 \\
        \midrule
        \multirow{2}{*}{\SI{65}{\nano\meter}}
            & 1:1 & \SI{64}{\micro\meter} & \SI{65}{\nano\meter} & --- & 300 & 400 \\
            & 1:n & --- & --- & --- & --- & --- \\
        \midrule
        \multirow{2}{*}{\SI{45}{\nano\meter}}
            & 1:1 & \SI{25}{\micro\meter}  & \SI{60}{\nano\meter}  & 26.0 & 200 & 250 \\
            & 1:n & \SI{320}{\micro\meter} & \SI{135}{\nano\meter} & 59.1 & 200 & 340 \\
        \bottomrule
    \end{tabularx}
    \label{tab:teo-vals-cs}
\end{table}

In the \SI{350}{\nano\meter} case, the $1{:}1$ configuration required an increase in $W$ to recover the target gain (near \SI{10}{\decibel}) due to initially low $g_m$. The load resistor was significantly reduced, which improved the NF with minimal gain impact. For the $1{:}n$ configuration, $W$ was optimized to maintain the scaling relation $n \cdot g_{mCG}$ while tuning $R$ for proper gain.

In the \SI{65}{\nano\meter} case, $W$ was adjusted to restore impedance matching and reduce NF. Lack of $1{:}n$ data suggests either infeasibility or omission in simulation.

In the \SI{45}{\nano\meter} case, initial attempts using $L = 3L_{\min}$ caused excessive capacitance, generating poles within the band of interest and degrading impedance flatness and gain. Switching to $L = L_{\min}$ mitigated this, but brought channel-length modulation effects. Final tuning of $W$, $L$, $g_m$, and $R$ ensured gain recovery and noise optimization.